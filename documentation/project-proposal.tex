\documentclass[a4paper,12pt]{article}
\usepackage{xcolor}
\usepackage{amsmath,amsfonts,amssymb}
\usepackage{geometry}
\usepackage{fancyhdr}
\usepackage{graphicx}
\usepackage{titlesec}
\usepackage{tikz}
\usepackage{booktabs}
\usepackage{array}
\usetikzlibrary{shadows}
\usepackage{tcolorbox}
\usepackage{float}
\usepackage{lipsum}
\usepackage{mdframed}
\usepackage{pagecolor}
\usepackage{mathpazo}   % Palatino font (serif)
\usepackage{microtype}  % Better typography

% Page background color
\pagecolor{gray!10!white}

% Geometry settings
\geometry{margin=0.5in}
\pagestyle{fancy}
\fancyhf{}

% Fancy header and footer
\fancyhead[C]{\textbf{\color{blue!80}CS754 Project Proposal}}
% \fancyhead[R]{\color{blue!80}Saksham Rathi}
\fancyfoot[C]{\thepage}

% Custom Section Color and Format with Sans-serif font
\titleformat{\section}
{\sffamily\color{purple!90!black}\normalfont\Large\bfseries}
{\thesection}{1em}{}

% Custom subsection format
\titleformat{\subsection}
{\sffamily\color{cyan!80!black}\normalfont\large\bfseries}
{\thesubsection}{1em}{}

% Stylish Title with TikZ (Enhanced with gradient)
\newcommand{\cooltitle}[1]{%
  \begin{tikzpicture}
    \node[fill=blue!20,rounded corners=10pt,inner sep=12pt, drop shadow, top color=blue!50, bottom color=blue!30] (box)
    {\Huge \bfseries \color{black} #1};
  \end{tikzpicture}
}
\usepackage{float} % Add this package

\newenvironment{solution}[2][]{%
    \begin{mdframed}[linecolor=blue!70!black, linewidth=2pt, roundcorner=10pt, backgroundcolor=yellow!10!white, skipabove=12pt, skipbelow=12pt]%
        \textbf{\large #2}
        \par\noindent\rule{\textwidth}{0.4pt}
}{
    \end{mdframed}
}

% Document title
\title{\cooltitle{CS754 Project Proposal}\\ FLINNG: Fast Locality-Sensitive Hashing for Approximate Near Neighbor Search via Group Testing}
\author{{\bf Saksham Rathi (22B1003), Ekansh Ravi Shankar (22B1032), Kshitij Vaidya (22B1829)}}
\date{}

\begin{document}
\maketitle
\section*{Description}
This project aims to implement the FLINNG (Filters to Identify Near-Neighbor Groups) algorithm for approximate near neighbor search as described in the NeurIPS 2021 paper. FLINNG combines group testing with locality-sensitive hashing (LSH) to provide a fast, memory-efficient solution for high-dimensional nearest neighbor search. The algorithm transforms the near neighbor search problem into a group testing problem by using distance-sensitive Bloom filters to identify groups containing near neighbors. Unlike traditional LSH approaches, FLINNG avoids expensive distance computations during queries and can be constructed in a single pass through the data.
The implementation will focus on building the core components of the FLINNG algorithm, including the group testing framework, distance-sensitive Bloom filters, and the threshold relaxation algorithm.

\section*{Objectives}
\begin{itemize}
  \item Implement the FLINNG algorithm for approximate near neighbor search as described in the paper
  \item Optimize the implementation for practical performance using the techniques outlined in Section 6
  \item Evaluate the algorithm's performance on real-world datasets and compare with baseline methods
  \item Analyze the trade-offs between query time, memory usage, and precision/recall
\end{itemize}

\section*{Datasets}
\begin{itemize}
  \item RefSeqG 
  \item RefSeqP 
  \item PromethION
  \item URL
  \item Webspam 
  \item YFCC100M
\end{itemize}

\end{document}
